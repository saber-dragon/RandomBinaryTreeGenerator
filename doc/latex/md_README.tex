\href{https://travis-ci.org/saber-dragon/RandomBinaryTreeGenerator}{\tt }

This repo implements a very simple random binary tree generator. The major purpose of this repo is to produce some test cases for binary-\/tree related problems on leetcode.

\subsection*{Usage}

see \href{./example.cpp}{\tt \char`\"{}example.\+cpp\char`\"{}}

\subsection*{Sample Output}

\begin{quote}
By {\ttfamily Simple\+Print}\+: \end{quote}

\begin{DoxyCode}
\{val : 6724, left : 9480, right : 223\}
\{val : 9480, left : 3885, right : 300\}
\{val : 3885, left : 8074, right : 9900\}
\{val : 8074, left : 9191, right : NULL\}
\{val : 9191, left : NULL, right : NULL\}
\{val : 9900, left : NULL, right : NULL\}
\{val : 300, left : 7552, right : NULL\}
\{val : 7552, left : NULL, right : 3938\}
\{val : 3938, left : NULL, right : NULL\}
\{val : 223, left : NULL, right : NULL\}
\end{DoxyCode}


\begin{quote}
By {\ttfamily Tree\+To\+Dot}\+: \end{quote}



\begin{DoxyCode}
digraph BinaryTree \{
node [shape = record,height=.1];
 node0[label = "<f0> |<f1> 6724|<f2> "];
 node1[label = "<f0> |<f1> 9480|<f2> "];
 node2[label = "<f0> |<f1> 3885|<f2> "];
 node3[label = "<f0> |<f1> 8074|<f2> "];
 node4[label = "<f0> |<f1> 9191|<f2> "];
 node5[label = "<f0> |<f1> 9900|<f2> "];
 node6[label = "<f0> |<f1> 300|<f2> "];
 node7[label = "<f0> |<f1> 7552|<f2> "];
 node8[label = "<f0> |<f1> 3938|<f2> "];
 node9[label = "<f0> |<f1> 223|<f2> "];
"node0":f0 -> "node1":f1;
"node1":f0 -> "node2":f1;
"node2":f0 -> "node3":f1;
"node3":f0 -> "node4":f1;
"node2":f2 -> "node5":f1;
"node1":f2 -> "node6":f1;
"node6":f0 -> "node7":f1;
"node7":f2 -> "node8":f1;
"node0":f2 -> "node9":f1;
\}
\end{DoxyCode}
 Note that the dot file can be viewed visualized with {\ttfamily xdot} or you can use {\ttfamily graphviz} to get the P\+NG by simple running {\ttfamily dot -\/\+Tpng -\/o example.\+png bt.\+dot}. The resulted P\+NG is shown below.

 